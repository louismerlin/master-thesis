%%%%%%%%%%%%%%%%%%%%%%%%%%%%%%%%%%%%%%%%%%%%%%%%%%%%%%%%%%%
% EPFL report package, main thesis file
% Goal: provide formatting for theses and project reports
% Author: Mathias Payer <mathias.payer@epfl.ch>
%
% This work may be distributed and/or modified under the
% conditions of the LaTeX Project Public License, either version 1.3
% of this license or (at your option) any later version.
% The latest version of this license is in
%   http://www.latex-project.org/lppl.txt
%
%%%%%%%%%%%%%%%%%%%%%%%%%%%%%%%%%%%%%%%%%%%%%%%%%%%%%%%%%%%
\documentclass[a4paper,11pt,oneside]{report}
% Options: MScThesis, BScThesis, MScProject, BScProject
\usepackage[MScThesis,lablogo]{EPFLreport}
\usepackage{xspace}

\title{Recovering type information from compiled binaries\\to aid in instrumentation}
\author{Louis Merlin}
\supervisor{Antony Vennard}
\adviser{Prof. Dr. sc. ETH Mathias Payer}
%\coadviser{Second Adviser}
\expert{Damian Pfammatter}

\newcommand{\sysname}{FooSystem\xspace}

\setlength{\parindent}{0cm}

\begin{document}
\maketitle
\makededication
\makeacks

\begin{abstract}
The abstract serves as an executive summary of your project.
Your abstract should cover at least the following topics, 1-2 sentences for
each: what area you are in, the problem you focus on, why existing work is
insufficient, what the high-level intuition of your work is, maybe a neat
design or implementation decision, and key results of your evaluation.
\end{abstract}

% \begin{frenchabstract}
% For a doctoral thesis, you have to provide a French translation of the
% English abstract. For other projects this is optional.
% \end{frenchabstract}

\maketoc

%%%%%%%%%%%%%%%%%%%%%%
\chapter{Introduction}
%%%%%%%%%%%%%%%%%%%%%%

% The introduction is a longer writeup that gently eases the reader into your
% thesis~\cite{dinesh20oakland}. Use the first paragraph to discuss the setting.
% In the second paragraph you can introduce the main challenge that you see.
% The third paragraph lists why related work is insufficient.
% The fourth and fifth paragraphs discuss your approach and why it is needed.
% The sixth paragraph will introduce your thesis statement. Think how you can
% distill the essence of your thesis into a single sentence.
% The seventh paragraph will highlight some of your results
% The eights paragraph discusses your core contribution.

% This section is usually 3-5 pages.


% Setting: compiled binaries; RetroWrite; [need for a compatible exchange format => DWARF;] closed-source C++ binaries in the wild (android?).

Work on C++ began in 1979, as a "C with classes"~\cite{cwithclasses}.
Since then, the language has grown in popularity, and even surpassed C itself~\cite{stackoverflowpopularity}.
Examples of well-known C++ projects include the zoom~\cite{zoom} conferencing software, the gold linker~\cite{gold} or [TODO: find another "well-known" example].
Nevertheless, reverse-engineering efforts have been focused towards C binaries, because analysis methods found this way will often work on C++ binaries too.


% Main challenge:
% C++-specific ELF sections are not as well documented as they could be (?);
% Debuggers do not work as well as they should on non-C binaries;

This has meant reverse engineering tools like ghidra~\cite{ghidra}, IDA Pro~\cite{ida} or Binary Ninja~\cite{binja} have treated non-C binaries as second-class citizen.
These tools will often show C++ specific features as passing comments, failing to show the real implications of a try/catch block or a polymorphic class.

% Ghidra/IDA will show as pseudo-C code, but not C++
%  see throw/catch or RTTI
% C++ is quite a compilicated language.
% C's abstractions work very well in assembler, but C++ adds a whole layer of complexity.
% When translated to assembly, a lot of information is lost.


% Related work:
% RetroWrite only does C;
% debugging tools don't have nice support for C++ features;
% DWARF will be explored more and more in the coming years;

The recent RetroWrite~\cite{dinesh20oakland} project by the HexHive lab is a static rewriting tool for x86\_64 position-independent binaries.
It enables the instrumentation of projects when we do not have access to the source code.
This can include a legacy project, a closed-source product or even malware.

% Does anything else try to do retrowriting C++ ?
% Does anything else try to extract classes ? (Marx / plugins)


% Our approach:
% Tool for static analysis of C++ binaries, to recover classes and display them in a helpful way;
% Fixing RetroWrite to support C++ binaries;

In this thesis we would like to present the \textbf{dis-cover}~\cite{discovergithub} static analysis tool,
as well as improvements made to RetroWrite to support C++ binaries.


% Why it is needed:
% Non-obtrusive way of adding C++ support in debugging tools;
% Nobody (?) (as far as we know / to the best of our knowledge) has been able to rewrite C++ binaries before, this could lead to very interesting discoveries;

The dis-cover tool is able to extract information from a C++ binary, and re-inject it as debug information using the DWARF format into the binary.
This enables other debugging tools to see and display this information.


% Highlight results:
% 53\% of packages on debian that use C++ have extracteable classes;
% [whatever we are able to do with C++ and RetroWrite];


% It's a first symbolic representation of C++ code that can be recompiled


% Core contribution:
% dis-cover as an open-source tool;
% [whatever we are able to add to RetroWrite];
% Also the documentation of the process.

[IF WE SUCCEEDED]
In this thesis we would like to show how we brought C++ support to RetroWrite, and what research opportunities this will create.

[IF WE FAILED]
In this thesis we will detail how we tried to bring C++ support to RetroWrite, and what remains to be done for the implementation to work.


%%%%%%%%%%%%%%%%%%%%
\chapter{Background}
%%%%%%%%%%%%%%%%%%%%

The background section introduces the necessary background to understand your
work. This is not necessarily related work but technologies and dependencies
that must be resolved to understand your design and implementation.

This section is usually 3-5 pages.

% C++ polymorphism (what, how, but also research like Marx) (+ exceptions) 
% also talk about how compilers will optimize this out in certain situtations
The C++ programming language implements polymorphism.
This feature enables complex code logic that can comply with external business logic for example.
Polymorphic classes are defined by having at least one virtual method, which is inheritable and overridable.
With this polymorphism comes type conversion, and more interestingly for us the \textbf{dynamic\_cast}~\cite{dynamiccast} expression.
Dynamic casting is a safe kind of type conversion.
It only will successfuly cast if the value is of the referenced type or a base type of that type.
In order to achieve that, the system must have some kind of information about the object's data type at runtime.
This information is called Run-Time Type Information (\textbf{RTTI}), and is stored in the binary if it is needed.
In certain situations, if there are polymorphic classes but certain features are not used (casting, inheritance logic), the classes are abstracted away at compile time and the RTTIs do not exist.
We will go into more details about the implementation of C++ RTTIs in later chapters.

% DWARF DIEs
\textbf{DWARF} is the debugging standard used widly in conjunction with executable ELF files.
It is often included in these ELF files in the \textbf{.debug\_info} section (and other related sections).
This debug information is made up of Debugging Information Entries (\textbf{DIEs}).
These entries can contain information about variable names, method definitions, the compilation process, or more importantly for us, class names and class inheritance.

% ELF sections

% How RetroWrite works


%%%%%%%%%%%%%%%%
\chapter{Design}
%%%%%%%%%%%%%%%%

Introduce and discuss the design decisions that you made during this project.
Highlight why individual decisions are important and/or necessary. Discuss
how the design fits together.

This section is usually 5-10 pages.

% Design => Decisions
% We decided not to reproduce Marx's work
% First attempt : recover classes from RTTI
% Future work : could take approaches like Marx and re-feed it like us

% VTable => RTTI
%   mention class data (debian analysis)

% Also: exceptions (we felt this needed to be handled)

% Why do we want DWARF data ?
%   We want to make the info available to reversing tools
%   => input from Dr. Sergey Bratus
%        He would like to see DWARF used as a lingua franca for debugging
% Also, exception's try/catch blocks & exception types
% but dwarf is limited for this (=> future work)

% Injecting debug section & symbols

% RetroWrite structure
%  x86_64 userspace and kernel, and arm_64
%  We've decided to focus on the x86_64 userspace
%    Most widely used target
%    Kernel drivers are mostly written in C
% We still want to keep the deterministic ability of retrowrite


%%%%%%%%%%%%%%%%%%%%%%%%
\chapter{Implementation}
%%%%%%%%%%%%%%%%%%%%%%%%

The implementation covers some of the implementation details of your project.
This is not intended to be a low level description of every line of code that
you wrote but covers the implementation aspects of the projects.

This section is usually 3-5 pages.

% Python script (because RetroWrite; and good tools) (1 sentence)

% RTTI and VTable

% Creating DIEs
% DIE Bytecode
% unwind tables etc
% what it does and does not do

% eu-unstrip etc

% RetroWrite implementation details
% Refer to the other thesis

%%%%%%%%%%%%%%%%%%%%
\chapter{Evaluation}
%%%%%%%%%%%%%%%%%%%%

In the evaluation you convince the reader that your design works as intended.
Describe the evaluation setup, the designed experiments, and how the
experiments showcase the individual points you want to prove.

This section is usually 5-10 pages.

% Small examples

% Big applications examples

% Debian test

% RetroWrite test


%%%%%%%%%%%%%%%%%%%%%%
\chapter{Related Work}
%%%%%%%%%%%%%%%%%%%%%%

The related work section covers closely related work. Here you can highlight
the related work, how it solved the problem, and why it solved a different
problem. Do not play down the importance of related work, all of these
systems have been published and evaluated! Say what is different and how
you overcome some of the weaknesses of related work by discussing the 
trade-offs. Stay positive!

This section is usually 3-5 pages.

% Marx


%%%%%%%%%%%%%%%%%%%%
\chapter{Conclusion}
%%%%%%%%%%%%%%%%%%%%

In the conclusion you repeat the main result and finalize the discussion of
your project. Mention the core results and why as well as how your system
advances the status quo.

\cleardoublepage
\phantomsection
\addcontentsline{toc}{chapter}{Bibliography}
\printbibliography

% Appendices are optional
% \appendix
% %%%%%%%%%%%%%%%%%%%%%%%%%%%%%%%%%%%%%%
% \chapter{How to make a transmogrifier}
% %%%%%%%%%%%%%%%%%%%%%%%%%%%%%%%%%%%%%%
%
% In case you ever need an (optional) appendix.
%
% You need the following items:
% \begin{itemize}
% \item A box
% \item Crayons
% \item A self-aware 5-year old
% \end{itemize}

\end{document}
