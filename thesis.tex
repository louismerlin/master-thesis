%%%%%%%%%%%%%%%%%%%%%%%%%%%%%%%%%%%%%%%%%%%%%%%%%%%%%%%%%%%
% EPFL report package, main thesis file
% Goal: provide formatting for theses and project reports
% Author: Mathias Payer <mathias.payer@epfl.ch>
%
% This work may be distributed and/or modified under the
% conditions of the LaTeX Project Public License, either version 1.3
% of this license or (at your option) any later version.
% The latest version of this license is in
%   http://www.latex-project.org/lppl.txt
%
%%%%%%%%%%%%%%%%%%%%%%%%%%%%%%%%%%%%%%%%%%%%%%%%%%%%%%%%%%%
\documentclass[a4paper,11pt,oneside]{report}
% Options: MScThesis, BScThesis, MScProject, BScProject
\usepackage[MScThesis,lablogo]{EPFLreport}
\usepackage{xspace}

\title{Recovering type information from compiled binaries\\to aid in instrumentation}
\author{Louis Merlin}
\supervisor{Antony Vennard}
\adviser{Prof. Dr. sc. ETH Mathias Payer}
%\coadviser{Second Adviser}
\expert{Damian Pfammatter}

\newcommand{\sysname}{FooSystem\xspace}

\begin{document}
\maketitle
\makededication
\makeacks

\begin{abstract}
The abstract serves as an executive summary of your project.
Your abstract should cover at least the following topics, 1-2 sentences for
each: what area you are in, the problem you focus on, why existing work is
insufficient, what the high-level intuition of your work is, maybe a neat
design or implementation decision, and key results of your evaluation.
\end{abstract}

% \begin{frenchabstract}
% For a doctoral thesis, you have to provide a French translation of the
% English abstract. For other projects this is optional.
% \end{frenchabstract}

\maketoc

%%%%%%%%%%%%%%%%%%%%%%
\chapter{Introduction}
%%%%%%%%%%%%%%%%%%%%%%

The introduction is a longer writeup that gently eases the reader into your
thesis~\cite{dinesh20oakland}. Use the first paragraph to discuss the setting.
In the second paragraph you can introduce the main challenge that you see.
The third paragraph lists why related work is insufficient.
The fourth and fifth paragraphs discuss your approach and why it is needed.
The sixth paragraph will introduce your thesis statement. Think how you can
distill the essence of your thesis into a single sentence.
The seventh paragraph will highlight some of your results
The eights paragraph discusses your core contribution.

This section is usually 3-5 pages.

Setting: compiled binaries; RetroWrite; push for DWARF; closed-source C++ binaries in the wild (android?).

Main challenge: C++-specific ELF sections are not as well documented as they could be (?).

Related work: RetroWrite only does C; debugging tools don't have nice support for C++ features; DWARF will be explored more and more in the coming years.

Our approach: Tool for static analysis of C++ binaries, to recover classes and display them in a helpful way. Fixing RetroWrite to support C++ binaries.

Why it is needed: Nobody (?) has been able to rewrite C++ binaries before, this could lead to very interesting discoveries.

Highlight results: 53\% of packages on debian that use C++ have extracteable classes, [whatever we are able to do with C++ and RetroWrite]

Core contribution: dis-cover as an open-source tool; [whatever we are able to add to RetroWrite]

%%%%%%%%%%%%%%%%%%%%
\chapter{Background}
%%%%%%%%%%%%%%%%%%%%

The background section introduces the necessary background to understand your
work. This is not necessarily related work but technologies and dependencies
that must be resolved to understand your design and implementation.

This section is usually 3-5 pages.


%%%%%%%%%%%%%%%%
\chapter{Design}
%%%%%%%%%%%%%%%%

Introduce and discuss the design decisions that you made during this project.
Highlight why individual decisions are important and/or necessary. Discuss
how the design fits together.

This section is usually 5-10 pages.


%%%%%%%%%%%%%%%%%%%%%%%%
\chapter{Implementation}
%%%%%%%%%%%%%%%%%%%%%%%%

The implementation covers some of the implementation details of your project.
This is not intended to be a low level description of every line of code that
you wrote but covers the implementation aspects of the projects.

This section is usually 3-5 pages.


%%%%%%%%%%%%%%%%%%%%
\chapter{Evaluation}
%%%%%%%%%%%%%%%%%%%%

In the evaluation you convince the reader that your design works as intended.
Describe the evaluation setup, the designed experiments, and how the
experiments showcase the individual points you want to prove.

This section is usually 5-10 pages.


%%%%%%%%%%%%%%%%%%%%%%
\chapter{Related Work}
%%%%%%%%%%%%%%%%%%%%%%

The related work section covers closely related work. Here you can highlight
the related work, how it solved the problem, and why it solved a different
problem. Do not play down the importance of related work, all of these
systems have been published and evaluated! Say what is different and how
you overcome some of the weaknesses of related work by discussing the 
trade-offs. Stay positive!

This section is usually 3-5 pages.


%%%%%%%%%%%%%%%%%%%%
\chapter{Conclusion}
%%%%%%%%%%%%%%%%%%%%

In the conclusion you repeat the main result and finalize the discussion of
your project. Mention the core results and why as well as how your system
advances the status quo.

\cleardoublepage
\phantomsection
\addcontentsline{toc}{chapter}{Bibliography}
\printbibliography

% Appendices are optional
% \appendix
% %%%%%%%%%%%%%%%%%%%%%%%%%%%%%%%%%%%%%%
% \chapter{How to make a transmogrifier}
% %%%%%%%%%%%%%%%%%%%%%%%%%%%%%%%%%%%%%%
%
% In case you ever need an (optional) appendix.
%
% You need the following items:
% \begin{itemize}
% \item A box
% \item Crayons
% \item A self-aware 5-year old
% \end{itemize}

\end{document}
